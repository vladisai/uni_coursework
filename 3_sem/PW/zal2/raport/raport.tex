\documentclass[a4paper]{article}
\author{Uladzislau Sobal ws374078}
\title{Raport dla zadania zaliczeniowego}
\usepackage[T1]{fontenc}
\usepackage{polski}
\usepackage[utf8x]{inputenc}
\usepackage{microtype}
%\usepackage{polyglossia}
%\setdefaultlanguage[babelshorthands]{polish}
%\usepackage[Ligatures=TeX]{fontspec}
\begin{document}
\maketitle
\section*{Implementacja i używane optymalizacje}

\subsection*{Kompresja wierzchołków}
Za pomocą hash mapy kompresowane są indeksy wszystkich wierzchołków żeby 
onie były w przedziale od 0 do N - 1, gdzie N - ilość wierzchołków. Dzięki 
tej kompresji, można używać zwykłych wektorów zamiast map 
przy obliczaniu. Za kompresję jest odpowiedzialna klasa CoordinateCompressor.

\subsection*{Podział pracy}
Wierzchołki dla których trzeba obliczyć wartości są przechowywane w wektorze importantVertices, z którego procesy biorą 
indeksy kiedy kończą swoje poprzednie zadanie za pomocą funkcji getNextVertexIndex.

\subsection*{Przekazywanie wyników}
Każdy proces wykonuje funkcje verticesProcessor. Ta funkcja dopóki może probuje otrzymać wierzchołek dla obrobienia, dla którego wywołuje funkcję processVertex, która wykonuje jedną iterację 
algorytma Brandesa dla danego wierzchołka. Żeby 
przyspieszyć działanie, ta funkcja przyjmuje jako argument wektor, do którego ona będzie dodawać wyniki.
W końcu, kiedy proces skończył pracę, on blokuje wektor wynika końcowego bc za pomocą muteksa bcMutex i zapisuje do bc swoje obliczone wartości.

\section*{Wyniki optymalizacji}
W tabeli poniżej podane są czasy wykonania oraz przyspieszenie na danych z wikipedii na komputerze students:
\\
\\
\begin{tabular}{|c|c|c|}
    \hline
    Ilość wątków & Czas (sekundy) & Przyspieszenie \\
    \hline
    1 & 12.7 & 1\\
    2 & 7.3 & 1.74\\
    3 & 4.33 & 2.933\\
    4 & 3.73 & 3.405\\
    5 & 3.16 & 4.019\\
    6 & 2.78 & 4.568\\
    7 & 2.19 & 5.799\\
    8 & 1.73 & 7.341\\
\hline


\end{tabular}

\end{document}

