\documentclass{sig-alternate-05-2015}

\begin{document}

% Copyright
\setcopyright{acmcopyright}
%\setcopyright{acmlicensed}
%\setcopyright{rightsretained}
%\setcopyright{usgov}
%\setcopyright{usgovmixed}
%\setcopyright{cagov}
%\setcopyright{cagovmixed}
\copyrtyr{2017}

% DOI
\doi{10.475/123_4}

% ISBN
\isbn{123-4567-24-567/08/06}

%Conference
\conferenceinfo{SOMIMUW '17}{May 22--June 1, 2017, Haifa, Israel}

%
% --- Author Metadata here ---
\conferenceinfo{SOMIMUW}{'17 Warsaw, Poland}
%\CopyrightYear{2007} % Allows default copyright year (20XX) to be over-ridden - IF NEED BE.
%\crdata{0-12345-67-8/90/01}  % Allows default copyright data (0-89791-88-6/97/05) to be over-ridden - IF NEED BE.
% --- End of Author Metadata ---

\title{Zero-blocks optimisation in MFS}

\numberofauthors{1} %  in this sample file, there are a *total*
% of THREE authors. 
%
\author{
\alignauthor
Uladzislau Sobal\\
       \affaddr{the University of Warsaw}\\
       \email{ws374078@students.mimuw.edu.pl}
}

\maketitle
\begin{abstract}
    This paper introduces and explains a simple optimisation for the Minix File System (MFS) that
    improves efficiency of storage for blocks filled with zeroes.
\end{abstract}

\section{Introduction}

Some files contain a lot of zeroes, so a natural question to ask would be if we can be more efficient
around these zeroes. To answer this question, let's get our hands dirty with the workings of MFS 
(and ext, since they are similar).

\section{Minix File System}
Both MFS and ext use i-nodes. I-node is a structure used to store data. It's basically an array of pointers.
The first 12 pointers are direct pointers to data.
The next 256 pointers are pointers to pointers to data (indirect blocks).
After that there are doubly indirect blocks (pointers to pointers to pointers)
and trebly indirect blocks (pointers to pinters to pointers to pointers).
What's interesting is if we are trying to read one of these blocks and it's not initialised
(the pointer is equal to 0), the system
just acts as if it was filled with zeroes. You can test this behaviour by creating a file, and 
dd'ing into it some random numbers with huge seek. 
Something like:

dd if=/dev/random bs=512 of=out seek=100000000

You will get a huge file that isn't even supposed to fit on the hard drive, and first $10^8$ blocks
will be filled with zeroes.

However, the system doesn't do anything of the kind when 
writing zeroes, it simply initialises a block and writes zeros. 
This is what I changed.

\section{Changing MFS source code}

To test the optimisation I altered standard MFS implementation.
The file that I changed was /fs/mfs/read.c.
There is a huge multi-purpose function called rw\_chunk. I altered so that
it first checks the data it is about to write, and, if the data is all zeroes,
doesn't create the block. Implementation is very simple, and the diff is only a couple
of dozens of lines long.

\section{Optimisation results}

To check how much we gain from this optimisation, I did 4 tests on different types of data -
code, books, films and minix image. 
I copied to a clean mfs and the patched one a folder containing code in
different languages, a folder with books and documents, a folder 
containing several episodes of a TV-series and .img file of a cloned system. Here are the results:

\begin{table}[ht]
\caption{Time}
\vspace{0.25cm}
\begin{tabular}{|l|c|c|}
    \hline
    Data type & Time without patch & Time with patch \\ 
    \hline
    Code & 16.226s & 16.142s \\
    TV series & 2m12.663s & 2m32.435s \\
    Documents & 10.542s & 11.973s \\
    minix.img & 9.593s & 13.733s \\
    \hline
\end{tabular}
\end{table}

\begin{table}[ht]
\caption{Size}
    \vspace{0.25cm}
\begin{tabular}{|l|c|c|}
    \hline
    Data type & Size without patch & Size with patch \\
    \hline
    Code & 219792 & 219328 \\
    TV series & 4029608 & 4029608 \\
    Documents & 249072 & 248920 \\
    minix.img & 373344 & 371408 \\
    \hline
\end{tabular}
\end{table}

All tests were run on a machine with CPU Intel i7-4510U, SSD.
Virtual machines were run with kvm enabled.

\section{Conclusion}

As you can see from the results, the optimisation is not always useful.
It does improve space usage a little bit, but it also takes more time, and the tradeoff is
not good enough to include the patch into the standard implementation.
However, if you have a special case where data
has a lot of zeroes, and time is not critical for you, you could use a patch that writes zeroes
more effectively. Since the data written in this way can be read by the unpatched version of the 
file system, the most sensible way to use the optimisation is 
to estimate how much you gain and run the patched write only when needed.

\bibliographystyle{abbrv}
\bibliography{sig-alternate-sample-systor2017}
\end{document}
